\documentclass[12pt, a4paper]{article}

\usepackage[utf8]{inputenc}
\usepackage{lmodern}
%\usepackage{fourier}
\usepackage{setspace}
	\singlespacing

\usepackage[frenchb]{babel}
\usepackage{xspace}
\usepackage[margin= 2.5cm]{geometry}
\pagestyle{plain}
\renewcommand{\thefootnote}{\fnsymbol{footnote}}

\usepackage{tikz}
	\usetikzlibrary{shapes}
\usepackage{graphicx}
	\graphicspath{{img/}}

\usepackage{varioref}
	\renewcommand{\reftextbefore}{page précédente}
	\renewcommand{\reftextfacebefore}{page ci-contre}
	\renewcommand{\reftextafter}{page suivante}
	\renewcommand{\reftextfaceafter}{page ci-contre}
	\renewcommand{\reftextcurrent}{}

\usepackage{amsmath, amsfonts}
\everymath{\displaystyle}


\newcommand{\espace}{\vspace{.8cm}}
\newcommand{\pg}{

}

%% REMPLIR
\usepackage[colorlinks=true, allcolors=blue, pdfborder={0 0 0}]{hyperref}
	\hypersetup{
		pdftitle={Logica Est},
		pdfsubject={Rapport Logica Est},
		pdfkeywords={Logica Est, IARISS, rapport},
		pdfauthor={IARISS Team}
	}
\title{Défi Logica Est}
\newcommand{\authors}{Florent et Nicolas}

%
\begin{document}

\author{\includegraphics{../_img/iariss_team.png} \\ {\sffamily \href{http://iarissteam.me}{iarissteam.me}}}
\date{\today}

\maketitle{}

{\sffamily Ce rapport a pour but de présenter les différents easter eggs implémentés dans notre application, ainsi que les astuces pour les découvrir, étant en rapport avec le thème Alsaco-Canadien}

\espace{}
Pour le respect du thème alsaco-canadien, nous avaons ajouté un thème disponible, qui change l'image de fond : nous avons, quand nous choisissons ce thème, un mélange d'un drapeau canadien et de cigogne, animal typique de la région Alsace.

Concernant les \emph{easter eggs}, le premier se déclenche à l'utilisation du \href{http://fr.wikipedia.org/wiki/Code_Konami}{code Konami}, soit la combinaison de touche haut haut bas bas gauche droite gauche droite B A.

Le deuxième se déclenche lorsque l'utilisateur tape dans la fenêtre du navigateur la chaine de caractère << je suis un caribou >>.

Il ne vous reste plus qu'à essayer pour voir ce que cela déclenche ! See \& enjoy !


\espace\vfill{}
Ce document a été rédigé en \LaTeX{} par \authors{} pour IarissTeam avec quelques tasses de café et beaucoup de bonne humeur.

Contactez-nous à \href{mailto:nuitinfo@iariss.com}{nuitinfo@iariss.com} pour tout renseignement supplémentaire !

\end{document}